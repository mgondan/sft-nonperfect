\documentclass[a4paper,12pt]{article}
\usepackage[utf8]{inputenc} 
\usepackage[english]{babel}
\usepackage{lmodern}
\usepackage{graphicx}
\usepackage{float}
\usepackage{amsmath}
\usepackage{fontenc}
%\usepackage{enumitem}
\usepackage{amssymb}
\usepackage{csquotes}
\usepackage{tikz}
\usepackage{physics}
\usetikzlibrary{matrix}
\usepackage[style=apa]{biblatex}


\title{Interaction contrast assuming a parallel exhaustive model} 
\author{Yiqi}
\date{\today}


\begin{document}
\maketitle

\vspace{24pt}

\section{Basic assumption of a parallel exhaustive model on RT}
\begin{align*}
T_{AB \mid r} = \max(D_{A \mid r},D_{B \mid r})
\end{align*}
\par

\section{Possible assumptions of a parallel exhaustive model on response}
There are several ways to determine the response assuming a parallel exhaustive model.

\subsection{The slower takes all}
\begin{align*}
R_{AB}=
\begin{cases}
C_A, & \text{ if } D_A>D_B \\
C_B, & \text{ if } D_A \leq D_B
\end{cases}
\end{align*}

I have been thinking about the psychological meaning of such a pattern. Under what circumstances should the final response determined only by the output of the slower process? Cannot think of a second example except ``report the identity of the second stimulus you perceive.''\par
It appear that the nature of the task should require a complete suppression of the faster stimulus (i.e., the faster is defined as distractor).

\subsection{All right}
\begin{align*}
R_{AB}=C_A \cdot C_B
\end{align*}
The final response is positive if and only if the outputs of both processes are positive. For example, the presence of a target should be reported if and only if it has been detected in both modalities.

\subsection{Screening}
\begin{align*}
R_{AB}=(1-C_A) \cdot (1-C_B)
\end{align*}
The final response is positive if and only if the outputs of both processes are negative. For example, the participant can report ``safe'' if and only if the target is absent in both modalities.

\subsection{XOR}
\begin{align*}
R_{AB}=C_A \cdot (1-C_B) + (1-C_A) \cdot C_B
\end{align*}
A positive response should be make if and only if the target appears exactly in one of the modalities. 

\subsection{Match}
\begin{align*}
R_{AB}=1-\big( C_A \cdot (1-C_B) + (1-C_A) \cdot C_B \big)
\end{align*}
A positive response should be make if and only if the stimuli perceived in both modalities match each other.

\par

\vspace{24pt}

\textbf{Remark}. Only the first pattern depends on which process is faster.

\section{IC under a parallel exhaustive model with perfect accuracy}
Since
$\max(D_{a \mid r}, D_{b \mid r}) \leq t \iff \big (D_{a \mid r} \leq t \cap D_{b \mid r} \leq t \big)$
, the inequality of IC for CDF can be derived directly.
\begin{align*}
& \big (\mathbb{P}(T_{ab} \leq t \mid r) + \mathbb{P}(T_{AB} \leq t \mid r) \big) - \big (\mathbb{P}(T_{aB} \leq t \mid r) + \mathbb{P}(T_{Ab} \leq t \mid r) \big) \\
=& \big(\mathbb{P}(D_a \leq t \mid r) \cdot \mathbb{P}(D_b \leq t \mid r) - \mathbb{P}(D_a \leq t \mid r) \cdot \mathbb{P}(D_B \leq t \mid r) \big) \\
& - \big(\mathbb{P}(D_A \leq t \mid r) \cdot \mathbb{P}(D_b \leq t \mid r) - \mathbb{P}(D_A \leq t \mid r) \cdot \mathbb{P}(D_B \leq t \mid r) \big) \\
=& \big(\mathbb{P}(D_a \leq t \mid r) - \mathbb{P}(D_A \leq t \mid r) \big) \cdot \big(\mathbb{P}(D_b \leq t \mid r) - \mathbb{P}(D_B \leq t \mid r) \big) \\
\geq & 0
\end{align*} 

\section{IC under a parallel exhaustive model with imperfect accuracy}


\end{document}